\usepackage[l2tabu,orthodox]{nag} % warn about bad habits :)
\usepackage[T1]{fontenc} % font encoding (allows hyphenation with umlaut)
\usepackage[utf8]{inputenc} % allow utf-8 characters
\usepackage{lmodern} % nice font
\usepackage{graphicx}
%\usepackage{epstopdf}
\usepackage[english]{babel} % hyphenation, bibliography, chapter, section, ...
\usepackage{datetime}
\usepackage{appendix}
\usepackage{stmaryrd}
\usepackage{rotating}
\usepackage{csquotes} % context sensitive quotes
\usepackage{amsmath,amsfonts,amssymb} % ams
\usepackage{amsthm} % enhanced \newtheorem
\usepackage{microtype} % nicer letter spacing
\usepackage{scrhack} % scrbook + float fix http://tex.stackexchange.com/questions/51867/koma-warning-about-toc
\allowdisplaybreaks % allow breaks in align, ...
\usepackage{soul}
% manual hyphenation (separated by spaces)
\hyphenation{
  ma-the-ma-tisch-en
  Schr\"o-ding-er
  Eli-sa-beth
}

% configure bibliography
\usepackage[backend=biber,safeinputenc,firstinits=true,doi=false,isbn=false,url=false,maxbibnames=5]{biblatex}
%\usepackage{biblatex}
\addbibresource{dissertation.bib}

% insert git revision
%\input{revision.tex}

\newcommand{\doctitle}{Iterative Solution of Discretized Convection-Diffusion~Problems}
\newcommand{\docsubtitle}{Theory and Practice}
% \newcommand{\doctitle}{Iterative Solution of Discretized Convection-Diffusion~Problems}
% \newcommand{\docsubtitle}{Theory and Practice}
\newcommand{\docauthor}{Carlos Echeverr{\'i}a Serur}
\newcommand{\docdate}{\currenttime}
% set title, subtitle, author, ... for titlepage
\title{\doctitle}
\subtitle{\docsubtitle}
\author{\docauthor}
\date{\docdate}

\usepackage{caption}
\usepackage[font=normalsize]{subcaption}
\renewcommand{\topfraction}{0.75}
\usepackage{pgfplots}
\pgfplotsset{compat=newest}
\usepgfplotslibrary{groupplots}
\usepackage{tikz}
\usetikzlibrary{matrix,arrows,shapes,positioning}
\usetikzlibrary{calc,tikzmark}
\usetikzlibrary{intersections}
\usetikzlibrary{patterns}
\usetikzlibrary{math} %needed tikz library
%\usetikzlibrary{decorations.markings}

\newcommand{\cross}{$\mathbin{\tikz [x=1.4ex,y=1.4ex,line width=.2ex, black] \draw (0,0) -- (1,1) (0,1) -- (1,0);}$}%
% \usetikzlibrary{decorations.markings}

% uncomment next 3 lines to disable caching of pgf plots
%\usepgfplotslibrary{external}
%\tikzexternalize[mode=list and make]{main}
%\tikzsetexternalprefix{cache/}




% -----------------------------------------------------------------------------
% maybe these can removed in the final version (smileys, skulls, ...)
% maybe you have to run pdflatex once instead of lualatex!
\usepackage{skull} % \skull
\usepackage{marvosym} % \Smiley{}
% 'debug' mode:
% \usepackage{todonotes} % \todo{}
%\usepackage{showkeys} % show ref keys
% \usepackage{refcheck} % check for numbered but unlabeled refs and unreferenced labels
% -----------------------------------------------------------------------------

\usepackage{bbm} % allow for double stroked small letters
\usepackage{MnSymbol}
\usepackage{color}
\usepackage{tabularx}
\usepackage{diagbox}
\usepackage{booktabs}
\usepackage{multirow} % cell spanning multiple rows
\usepackage{xstring}
\usepackage{ifthen}
\usepackage{mathdots} % nicer ddots in matrices
\usepackage{enumitem} % resumed enumerates
\setlist[enumerate,2]{ref=\theenumi.\alph*)} % ref nested like 2.a)

\usepackage{ccicons} % creative commons icons
%\usepackage{titling} % custom title page
\usepackage{numprint} % pretty print numbers
\usepackage{varwidth} % center itemize/description

\usepackage{algorithm} % algorithm floating env
\usepackage{algpseudocode} % algorithmicx
\renewcommand{\algorithmicrequire}{\textbf{Input:}}
\newcommand{\pushcode}[1][1]{\hskip#1\dimexpr\algorithmicindent\relax}
\newcommand{\algindent}[1]{\dimexpr\algorithmicindent*#1\relax}
\newcommand{\algwidth}[1]{\dimexpr\linewidth-\algorithmicindent*#1\relax}
\newcommand{\algmulti}[2][]{\parbox[t]{\algwidth{#1}}{#2\strut}}

\usepackage{listings} % actual code
\lstset{
  language=Python, % Python, Python, Python!
  basicstyle=\ttfamily, % monospaced font
  breaklines=true, % break lines
  keywordstyle=\color{blue},
  commentstyle=\color{gray},
  xleftmargin=0.5cm,
}

% http://tex.stackexchange.com/questions/33979/include-a-line-break-in-algorithmic-while-maintaining-indentation
%\usepackage{varwidth} % for multilines in algorithms (like minipages)

\usepackage[normalem]{ulem} % used for strikeout \sout
% The bm package defines a command \bm which makes its argument bold.
% The argument may be any maths object from a single symbol to an expression.
% This is closely related to the specification of the \boldsymbol command in
% AMS-LaTeX, but \bm is rather more careful in the way it does things.
\newcommand\bmmax{2}
\usepackage{bm}
\usepackage{calrsfs}
\DeclareMathAlphabet{\mathscr}{OMS}{zplm}{m}{n}
\usepackage{chngcntr} % number figures, algorithms by chapter, e.g., Algo. 2.1
\counterwithin{figure}{chapter}
\counterwithin{algorithm}{chapter}
\counterwithin{table}{chapter}

\usepackage{hyperxmp} % metadata for pdf/a-1b
\usepackage[pdfa]{hyperref} % load as last package! contains url package
\hypersetup{
  draft=false,
  pdftitle={\doctitle~--~\docsubtitle},
  pdfauthor={\docauthor},
  pdfcontactemail={echeverriacarlos@gmail.com},
  pdfcopyright={Copyright (C) 2017, \docauthor.
    This work is licensed under the Creative Commons Attribution-ShareAlike 4.0
    International License (CC-BY-SA 4.0).},
  pdflicenseurl={http://creativecommons.org/licenses/by-sa/4.0/deed.en\_US},
  pdfkeywords={Dissertation, mathematics, numerical analysis, Krylov,
    linear system, nonnormal, convection-diffusion, multiplicative Schwarz},
  pdfsubject={Dissertation of \docauthor on the iterative solution of nonnormal linear algebraic systems arising in convection-diffusion problems.},
  pdflang={en},
  colorlinks=true,
  linkcolor=black,
  citecolor=blue,
  filecolor=cyan,
  urlcolor=red
}

% embed color profile
% cf. http://ioctl.eu/blog/2009/02/21/pdflatex_colorprofiles/
% and http://tex.stackexchange.com/questions/95792/device-independent-color
%\immediate\pdfobj stream attr{/N 4} file{eci_rgb_profile/eciRGB_v2.icc}
%\pdfcatalog{%
%/OutputIntents [ <<
%/Type /OutputIntent
%/S/GTS_PDFA1
%/DestOutputProfile \the\pdflastobj\space 0 R
%/OutputConditionIdentifier (CGATS TR001)
%>> ]
%}

% uppercase cali: operators on infinite-dimensional spaces
\newcommand\operatorinf[1]{\ensuremath{\mathsf{#1}}}
% uppercase bold: operators on finite-dimensional spaces (also matrices)
\newcommand\operatorfin[1]{\ensuremath{\mathbf{#1}}}
% lowercase: vector
\newcommand\vect[1]{\ensuremath{#1}}
% uppercase: tuples of vectors
\newcommand\vecttuple[1]{\ensuremath{#1}}
% uppercase: vector-spaces
\newcommand\vectspace[1]{\ensuremath{\mathcal#1}}
% input autogenerated commands
%\input{autogen_commands.tex}

% number sets
\newcommand\N{\ensuremath{\mathbb{N}}}
\renewcommand\Re{\ensuremath{\mathbb{R}}}
\newcommand\Co{\ensuremath{\mathbb{C}}}
\newcommand\Polys{\ensuremath{\mathbb{P}}}


% other helpers
\newcommand\DEF{\ensuremath{\mathrel{\mathop:}=}}
\newcommand\FED{\ensuremath{\mathrel{=\mathop:}}}
\newcommand\igralnl[4]{\ensuremath{\int\nolimits_{#1}^{#2} #3 \, \mathrm{d} #4}}
\newcommand\iu{\ensuremath{\mathbbm{i}}} % imaginary unit
\newcommand\e{\ensuremath{\mathrm{e}}} % Euler's number
\newcommand\clos[1]{\ensuremath{\overline{#1}}} % closure
\newcommand\conj[1]{\ensuremath{\overline{#1}}} % complex conjugate
\newcommand\tp{\ensuremath{\mathsf{T}}} % transpose
\newcommand\htp{\ensuremath{\mathsf{H}}} % hermitian transpose
\newcommand\inv{\ensuremath{{-1}}} % inverse
\newcommand\adj{\ensuremath{{\star}}} % adjoint
\newcommand\idx{\ensuremath{{(i)}}}
\newcommand\til[1]{\ensuremath{\widetilde{#1}}}
\newcommand\wh[1]{\ensuremath{\widehat{#1}}}
\newcommand\restr[2]{\ensuremath{\left.{#1}\right|_{#2}}}
\newcommand\id{\ensuremath{\mathsf{id}}}
\newcommand\zr{\ensuremath{\mathsf{0}}}
\newcommand{\mat}[1]{\begin{bmatrix}#1\end{bmatrix}}
\newcommand\LRA{\ensuremath{\Longrightarrow}}
\newcommand\lra{\ensuremath{\longrightarrow}}
\newcommand\LLRA{\ensuremath{\Longleftrightarrow}}
\newcommand\llra{\ensuremath{\longleftrightarrow}}
\newcommand\RA{\ensuremath{\Rightarrow}}
\newcommand\ra{\ensuremath{\rightarrow}}
\newcommand{\Span}[1]{\ensuremath\lsem{#1}\rsem}
\newcommand\eff{\ensuremath{\text{eff}}}
\newcommand\CG{\ensuremath{\text{CG}}} % conjugate gradient
\newcommand\MR{\ensuremath{\text{MR}}} % minimal residual
\newcommand\Laplace{\ensuremath{\Delta}}
\newcommand\PetrovAndOrGalerkin{\mbox{(Petrov--)}\allowbreak Galerkin}
\newcommand\vsV{\ensuremath{\mathcal}}
\newcommand{\eps} {\varepsilon}

% inner product
%  \ip{x}{y}    results in <x,y>
%  \ip[M]{x}{y} results in <x,y>_M
\newcommand{\ip}[3][]{\ensuremath{{\left\langle{#2},{#3}\right\rangle}_{#1}}}
% ip dots
\newcommand{\ipdots}[1][]{\ensuremath{{\left\langle{\cdot},{\cdot}\right\rangle}_{#1}}}
% norm
\newcommand{\nrm}[2][]{\ensuremath{\left\|#2\right\|_{#1}}}
\newcommand{\nrmdots}[1][]{\ensuremath{\left\|\cdot\right\|_{#1}}}
% 2-norm
\newcommand{\nrmeuc}[1]{\nrm[2]{#1}}
% norm induced by inner product
\newcommand{\nrmip}[1]{\nrm[\ipdots]{#1}}
% absolute value
\newcommand{\abs}[1]{\left|#1\right|}

% input experiment
%\newcommand{\inputplot}[1]{\input{figures/#1.tikz}}
% \protect is from http://www.latex-community.org/forum/viewtopic.php?f=5&t=12497
%\newcommand{\inputraw}[1]{\protect\input{raw/#1}}

\newcommand{\atopfrac}[2]{\genfrac{}{}{0pt}{}{#1}{#2}}


\newcommand{\cred}[1]{{\color{red}{#1}}}
\newcommand{\cblue}[1]{{\color{blue}{#1}}}

\newcommand{\klein}[1]{\scriptscriptstyle #1}
\newcommand{\entryvE}{\ensuremath{b_{\klein{H}}}}
\newcommand{\entryvW}{\ensuremath{c_{\klein{H}}}}
\newcommand{\entryvC}{\ensuremath{a_{\klein{H}}}}
\newcommand{\entryvN}{\ensuremath{e_{\klein{H}}}}
\newcommand{\entryvS}{\ensuremath{d_{\klein{H}}}}
\newcommand{\entrywE}{\ensuremath{b_{\klein{h}}}}
\newcommand{\entrywW}{\ensuremath{c_{\klein{h}}}}
\newcommand{\entrywC}{\ensuremath{a_{\klein{h}}}}
\newcommand{\entrywN}{\ensuremath{e_{\klein{h}}}}
\newcommand{\entrywS}{\ensuremath{d_{\klein{h}}}}
\newcommand{\entryzE}{\ensuremath{b}}
\newcommand{\entryzW}{\ensuremath{c}}
\newcommand{\entryzC}{\ensuremath{a}}
\newcommand{\dsd}{\ensuremath{d}}
\newcommand{\dsH}{\ensuremath{d_{\klein{H}}}}
\newcommand{\dsh}{\ensuremath{d_{\klein{h}}}}
\newcommand{\entryzN}{\ensuremath{e}}
\newcommand{\entryzS}{\ensuremath{d}}
\newcommand{\matAH}{\ensuremath{\mathbf{A}_{\klein{H}}}}
\newcommand{\matAHhat}{\ensuremath{\widehat{\mathbf{A}}_{\klein{H}}}}
%\newcommand{\matBH}{\ensuremath{\hat{B}}}
\newcommand{\matBH}{\ensuremath{\mathbf{B}_{\klein{H}}}}
\newcommand{\matCH}{\ensuremath{\mathbf{C}_{\klein{H}}}}
\newcommand{\matAh}{\ensuremath{\mathbf{A}_{\klein{h}}}}
\newcommand{\matAhhat}{\ensuremath{\widehat{\mathbf{A}}_{\klein{h}}}}
\newcommand{\matBh}{\ensuremath{\mathbf{B}_{\klein{h}}}}
\newcommand{\matCh}{\ensuremath{\mathbf{C}_{\klein{h}}}}
%\newcommand{\matCh}{\ensuremath{\tilde{C}}}
\newcommand{\matA}{\ensuremath{\widehat{\mathbf{A}}}}
%\newcommand{\suAs}{\ensuremath{\widetilde{A}}}
\newcommand{\matB}{\ensuremath{\mathbf{B}}}
\newcommand{\matC}{\ensuremath{\mathbf{C}}}
\newcommand{\matSH}{\ensuremath{\mathbf{S}_{\klein{H}}}}
\newcommand{\matSh}{\ensuremath{\mathbf{S}_{\klein{h}}}}
\newcommand{\matSs}{\ensuremath{\mathbf{S}}}
\newcommand{\matNH}{\ensuremath{\mathbf{N}_{\klein{H}}}}
\newcommand{\matNh}{\ensuremath{\mathbf{N}_{\klein{h}}}}
\newcommand{\matNs}{\ensuremath{\mathbf{N}}}
\newcommand{\piOne}{\ensuremath{\pi^{\klein{(1)}}}}
\newcommand{\piTwo}{\ensuremath{\pi^{\klein{(2)}}}}
\newcommand{\pOne}{\ensuremath{\mathbf{p}^{\klein{(1)}}}}
\newcommand{\pTwo}{\ensuremath{\mathbf{p}^{\klein{(2)}}}}
\newcommand{\PiOne}{\ensuremath{\mathbf{\Pi}^{\klein{(1)}}}}
\newcommand{\PiTwo}{\ensuremath{\mathbf{\Pi}^{\klein{(2)}}}}
\newcommand{\POne}{\ensuremath{\mathbf{P}^{\klein{(1)}}}}
\newcommand{\PTwo}{\ensuremath{\mathbf{P}^{\klein{(2)}}}}
\newcommand{\pI}{\ensuremath{p^{\klein{(i)}}}}
\newcommand{\pII}{\ensuremath{\mathbf{p}^{\klein{(i)}}}}
\newcommand{\Ubar}{\overline{U}}
\newcommand{\pe}{\psi}
\newcommand{\Tr}{\ensuremath{\klein{T}}}

% vector quantities
\newcommand\A{\ensuremath{\mathbf{A}}}
\newcommand\B{\ensuremath{\mathbf{B}}}
\newcommand\C{\ensuremath{\mathbf{C}}}
\newcommand\D{\ensuremath{\mathbf{D}}}
\newcommand\E{\ensuremath{\mathbf{E}}}
\newcommand\F{\ensuremath{\mathbf{F}}}
\newcommand\G{\ensuremath{\mathbf{G}}}
\renewcommand\H{\ensuremath{\mathbf{H}}}
\newcommand\I{\ensuremath{\mathbf{I}}}
\renewcommand\L{\ensuremath{\mathbf{L}}}
\newcommand\M{\ensuremath{\mathbf{M}}}
\renewcommand\N{\ensuremath{\mathbf{N}}}
\renewcommand\O{\ensuremath{\mathbf{O}}}
\renewcommand\P{\ensuremath{\mathbf{P}}}
\newcommand\Q{\ensuremath{\mathbf{Q}}}
\newcommand\R{\ensuremath{\mathbf{R}}}
% \renewcommand\S{\ensuremath{\mathbf{S}}}
\newcommand\T{\ensuremath{\mathbf{T}}}
\newcommand\U{\ensuremath{\mathbf{U}}}
\newcommand\V{\ensuremath{\mathbf{V}}}
\newcommand\W{\ensuremath{\mathbf{W}}}
\newcommand\X{\ensuremath{\mathbf{X}}}
\newcommand\Y{\ensuremath{\mathbf{Y}}}
\newcommand\Z{\ensuremath{\mathbf{Z}}}
\renewcommand\b{\ensuremath{\mathbf{b}}}
\renewcommand\e{\ensuremath{\mathbf{e}}}
\newcommand\f{\ensuremath{\mathbf{f}}}
\newcommand\n{\ensuremath{\bm{n}}}
\renewcommand\r{\ensuremath{\mathbf{r}}}
\renewcommand\u{\ensuremath{\mathbf{u}}}
\renewcommand\v{\ensuremath{\mathbf{v}}}
\newcommand\x{\ensuremath{\mathbf{x}}}
\newcommand\y{\ensuremath{\mathbf{y}}}
\newcommand\om{\ensuremath{\bm{\omega}}}
\newcommand\xib{\ensuremath{\bm{\xi}}}


% -----------------------------------------------------------------------------
\DeclareMathOperator{\spn}{span}
\DeclareMathOperator{\diag}{diag}
\DeclareMathOperator{\range}{range}
\DeclareMathOperator{\argmin}{argmin}
\DeclareMathOperator{\In}{In}
\DeclareMathOperator{\rank}{rank}
\DeclareMathOperator{\Real}{Re}
\DeclareMathOperator{\Imag}{Im}
\DeclareMathOperator{\AMG}{AMG}
\DeclareMathOperator{\tridiag}{tridiag}

% -----------------------------------------------------------------------------
\newlength\figurewidth
\newlength\figureheight

% -----------------------------------------------------------------------------
\newtheorem{thm}{Theorem}[chapter] % numbering by chapter, e.g., Theorem 2.12
\newtheorem{prop}[thm]{Proposition}
\newtheorem{lemma}[thm]{Lemma}
\newtheorem{cor}[thm]{Corollary}
\newtheorem{example}[thm]{Example}%[section]
% \theoremstyle{definition}
\newtheorem{definition}[thm]{Definition}
\newtheorem{rmk}[thm]{Remark}
\newtheorem{ex}[thm]{Example}
\newtheorem{ass}[thm]{Assumption}
\newtheorem{setup}[thm]{Setup}



% customize dictum format:
\setkomafont{dictumtext}{\itshape\small}
\setkomafont{dictumauthor}{\normalfont}
\renewcommand*\dictumwidth{\linewidth}
\renewcommand*\dictumauthorformat[1]{--- #1}
\renewcommand*\dictumrule{}

\newcommand{\td}[1]{\todo[color=green!40]{#1}}

% \newenvironment{poliabstract}[1]
%    {\renewcommand{\abstractname}{#1}\begin{abstract}}
%    {\end{abstract}}


\newenvironment{changemargin}[2]{%
\begin{list}{}{%
\setlength{\topsep}{0pt}%
\setlength{\leftmargin}{#1}%
\setlength{\rightmargin}{#2}%
\setlength{\listparindent}{\parindent}%
\setlength{\itemindent}{\parindent}%
\setlength{\parsep}{\parskip}%
}%
\item[]}{\end{list}}


\usepackage{afterpage}

\newcommand\blankpage{%
    \null
    \thispagestyle{empty}%
    \addtocounter{page}{-1}%
    \newpage}

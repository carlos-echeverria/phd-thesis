%
% Let the positve real number $J\geq 3$ represent the number of intervals used in
% the homogenous mesh in the $x$-direction, and the even positive integer
% $N\geq 4$ denote the number of mesh intervals used for the Shishkin mesh in
% the $y$-direction. We define the transition parameter $\tau_y$, which will be
% used to specify where the mesh changes from coarse to fine in, by
% %
% \begin{equation}\label{eq:tau}
% \tau_y \equiv \min\left\{\frac12,2\frac{\epsilon}{\alpha_2}  \ln N\right\}.
% \end{equation}
% %
% The assumption $\alpha_2\gg \epsilon$ (equivalently $\epsilon \leq CN^{-1}$)
% leads to the value $\tau_y=2\frac{\epsilon}{\alpha_2}\ln N\ll 1$,
% so that the \emph{mesh transition point} $1-\tau_y$ will be very close to $y=1$.
% The use of this hybrid scheme will divide the domain $\Omega$ into two
% overlapping subdomains, $\overline{\Omega}=\Omega_{1}\cup\Omega_{2}$, where
% %
% \[
% \Omega_{1}=[0,1]\times[0,1-\tau_y],\quad\Omega_{2}=[0,1]\times[1-\tau_y,1],
% \]
% %
% this subdivision is shown in the left side of Figure~\ref{fig:2Ddomain2}.
% %
% \begin{figure}[h!]
% \hspace*{-2em}
% \begin{minipage}[b]{0.5\textwidth}
% \begin{center}
% \begin{tikzpicture}[scale = 0.4]
% %%%% Rectangle:
% \draw (0,0) rectangle (10, 10);
% %%%% horizontal lines and $y$
% \node[left] at (0,10) {$1$};
% \draw (0,8) node[left] {$1-\tau_y$} -- (10,8);
% \node[left] at (0,0) {$0$};
% \node[below] at (0,0) {$0$};
% \node[below] at (10,0) {$1$};
% %%%% Black dots:
% \draw[black,fill=black] (0,8) circle (0.5ex);
% %%%% Tile labels
% \node at (5,4) {$\Omega_1$};
% \node at (5,9) {$\Omega_2$};
% \end{tikzpicture}
% \end{center}
% \end{minipage}
% %
% \hspace*{1em}
% %
% \begin{minipage}[b]{0.5\textwidth}
% \begin{center}
% \begin{tikzpicture}[scale = 0.4]
% %%%% Rectangle:
% \draw (0,0) rectangle (10, 10);
%
% %%%% horizontal lines and $y$
% \foreach \point in {2,4,6,8,8.5,9,9.5}
% \draw (0,\point) -- (10,\point);
% \node[left] at (0,10) {$y_N$};
% \node[left] at (0,8)  {$y_n$};
% \node[left] at (0,0)  {$y_0$};
%
% %%%% vertical lines and $x$:
% \foreach \point in {2,4,6,8}
% \draw (\point,0) -- (\point,10);
% \node[below] at (0,0) {$x_0$};
% \node[below] at (10,0) {$x_J$};
%
% %%%% Black dots:
% \draw[black,fill=black] (0,8) circle (0.5ex);
% %%%% Tile labels
% \draw (5,10) node[above] {$H_x$};
% \draw (10,2.5) node[right] {$H_y$};
% \draw (10,8.7) node[right] {$h_y$};
% \end{tikzpicture}
% \end{center}
% \end{minipage}
% %\includegraphics[width=0.9\linewidth]{figures/simpleShish2D3.pdf}
% \caption{Division of the domain and Shishkin mesh for equation \eqref{eq:2D:bvp2}
% with one outflow exponential layer.}
% \label{fig:2Ddomain2}
% \end{figure}
%
% Let $n\equiv \frac{N}{2}$, if we denote the mesh width in the $x$-direction
% by $H_x$ and the mesh widths inside and outside the boundary layer in the
% $y$-direction by $h_y$ and $H_y$, i.e.,
% %
% \begin{equation}\label{eq:nHh}
% H_x:=\frac{1}{J},
% \;\;\quad
% h_y:=\frac{\tau_y}{n},
% \;\;\quad\mbox{and}\;\;\quad
% H_y:=\frac{( 1-\tau_y)}{n},
% \end{equation}
% then the nodes of the mesh
% $
% \{(x_i,y_j)\in\overline{\Omega}:i=0,\dots,J; j=0,\dots,N\}
% $
% are given by
% %
% \[x_i\equiv iH_x,
% \quad \mbox{ and } \quad
% y_j\equiv
% \begin{cases} jH_y &\mbox{for}\;\; j=0,\ldots,n, \\
% 1-(N-j)h_y &\mbox{for}\;\; j=n+1,\ldots,N. \end{cases}\]
% %
% %The mesh is constructed by drawing lines parallel to the coordinate axes
% %through these mesh points; see the right side of Figure~\ref{fig:2Ddomain}.
% Here $x_0=y_0=0$ and $x_J=y_N=1$ so that the mesh consists of $K=J-1$ interior
% nodes in the $x$-direction and $M:=N-1$ interior nodes in the $y$-direction,
% where the node $y_n$ is exactly at the transition point $1-\tau_y$.
% It is clear that the mesh widths on $\Omega_{1}$ satisfy $1/J\leq H_x\leq 2/J$
% and $1/N\leq H_y\leq 2/N$ so the mesh is coarse in this domain. On the other
% hand, $h_y$ is $\mathscr{O}(\epsilon N^{-1}\log(N))$, so on $\Omega_2$ the mesh
% is coarse in the $x$ direction and fine in the $y$ direction.  The ratio
% between the different mesh sizes in the $y$ direction is
% %
% \[\frac{h_y}{H_y}=\frac{\tau_y}{1-\tau_y}=\tau_y+{\cal O}(\tau_y^2) \ll 1.\]

% Using the standard upwind finite difference operators and the lexicographical
% line ordering of the unknowns of the hybrid mesh, the finite difference scheme
% yields a linear algebraic system $\A\u=\f$, \td{\textbf{fix}: change to the used
% notation of our generic linear system} with a coefficient matrix $\A$ of the
% form \eqref{eq:2D:blockmat} and where the submatrices $\matAHhat$ and
% $\matAhhat$ have the block tridiagonal structure
% \eqref{eq:2D:tridiag}.
% The blocks in \eqref{eq:2D:tridiag} are of dimension $M\times M$ where
% $M\equiv N-1$. The off-diagonal blocks are multiples of the identity matrix $\I\in{\mathbb R}^{M\times M}$,
% $\C_H = d_H \I$, ${\C} = d \I$, $\C_h = d_h \I$,
% $\B_H = e_H \I$, ${\B} = e \I$, $\B_h = e_h \I$,
% where
% \begin{align*}
% 	&\entryvS  =  - \frac{\epsilon}{H_y^{2}}-\frac{\alpha_2}{H_y},&
% 	&\entryzS  =  - \frac{2\epsilon}{H_y(H_y+h_y)}-\frac{\alpha_2}{H_y},&
% 	&\entrywS  =  - \frac{\epsilon}{h_y^{2}}-\frac{\alpha_2}{h_y},\\
%     &\entryvN  =  - \frac{\epsilon}{H_y^{2}},&
%     &\entryzN  =  - \frac{2\epsilon}{h_y(H_y+h_y)},&
%     &\entrywN  =  - \frac{\epsilon}{h_y^{2}}.
% \end{align*}
% and $\A_H=\text{tridiag}(\entryvW,\entryvC,\entryvE)$, ${\matA}=\text{tridiag}(\entryzW,\entryzC,\entryzE)$, $\A_h=\text{tridiag}(\entrywW,\entrywC,\entrywE)$, where
% \begin{align*}
% \hspace{-3em}
%     &\entryvW  =  -\frac{\epsilon}{H_x^{2}},& % - \frac{\alpha_1}{H_x},&
%     &\entryvC  =  \frac{2\epsilon}{H_x^{2}} +\frac{2\varepsilon}{H_y^{2}}
% %    +\frac{\alpha_1}{H_x}
%     +\frac{\alpha_2}{H_y}+\beta,
%     &\entryvE  =  -\frac{\epsilon}{H_x^{2}},  \\
%     &\entryzW  =  -\frac{\epsilon}{H_x^{2}} ,&
% %    - \frac{\alpha_1}{H_x},&
%     &\entryzC  =  \frac{2\epsilon}{H_x^{2}} +\frac{2\varepsilon}{H_yh_y}
%     %+\frac{\alpha_1}{H_x}
%     +\frac{\alpha_2}{H_y}+\beta,
%     &\entryzE  =  -\frac{\epsilon}{H_x^{2}}, \label{eq:2D:upwind}\\
%     &\entrywW  =  -\frac{\epsilon}{H_x^{2}},& %- \frac{\alpha_1}{H_x},&
%     &\entrywC  =  \frac{2\epsilon}{H_x^{2}} +\frac{2\varepsilon}{h_y^{2}}
%     %+\frac{\alpha_1}{H_x}
%     +\frac{\alpha_2}{h_y}+\beta,
%     &\entrywE  =  -\frac{\epsilon}{H_x^{2}}.
% \end{align*}
% Note that the off-diagonal entries are negative, and the diagonal entries $a_H$, $a$, $a_h$ are positive. Moreover,
% $$
% a_H + b_H + c_H + d_H + e_H
% = a + b + c + d + e
% = a_h + b_h + c_h + d_h + e_h = \beta \geq 0
% $$
%
% Considering the matrix $\infty$-norm we will show that
% the assumptions of Theorem~\ref{thm:2D:main2} are satisfied.
% In more detail, we will prove that
% $\A$ satisfies \eqref{eq:2D:dominant}, i.e., it is
% row block diagonally dominant, and that
% the matrices $\matAHhat$ and $\matAhhat$ satisfy the conditions
% \eqref{eq:2D:dominant2}, i.e., they are column block diagonally dominant.
% Since all considered blocks are nonsingular, Theorem~\ref{thm:2D:main2} can then be applied.
%
% To prove \eqref{eq:2D:dominant} and \eqref{eq:2D:dominant2} we just need to show that
% \begin{equation}\label{eq:2D:shishkinblocks}
%  |e_H+d_H| \|\matAH^{-1}\|_\infty \leq 1,\quad
% |e+d| \|\matA^{-1}\|_\infty \leq 1,\quad
% |e_h+d_h| \|\matAh^{-1}\|_\infty \leq 1,
% \end{equation}
% i.e., we need to bound the $\infty$-norms of matrices
% $\matAH^{-1}$, $\matA^{-1}$, and $\matAh^{-1}$.
%
%
% First realize
% that for any \emph{nonsingular} matrix
% $\mathscr{M}$ and an induced matrix norm it holds that
% \[
% \|\mathscr{M}^{-1}\|=\max_{\|v\|=1}\left\Vert \mathscr{M}^{-1}\left(\frac{\mathscr{M}v}{\|\mathscr{M}v\|}\right)\right\Vert =\frac{1}{\min_{\|v\|=1}\|\mathscr{M} v\|}.
% \]
% Therefore, if $\|\mathscr{M} v\|\geq\gamma>0$ for any unit norm vector $v$,
% then $\|\mathscr{M}^{-1}\|\leq{\gamma^{-1}}$.
%
%
% Second, suppose that $\mathscr{M}$ is a strictly diagonally dominant
% tridiagonal Toeplitz matrix
% $\mathscr{M}=\mathrm{tridiag}(\hat{c},\hat{a},\hat{b})$,
% where $\hat{a}>0$, $\hat{b}<0$, $\hat{c}<0$, and
% $
% \hat{a}+\hat{b}+\hat{c}>0.
% $
% We would like to bound $\|\mathscr{M} v\|_{\infty}$ for any unit norm vector $v$
% from below. If $\|v\|_{\infty}=1$, then there is an index $i$ such that
% $|v_{i}|=1$. Without loss of generality we can assume that $v_{i}=1$, because
% changing the sign of the vector does not change $\|\mathscr{M} v\|_{\infty}$.
% Defining $v_{0}=0$ and $v_{n+1}=0$, it holds that
% \[
% \|\mathscr{M} v\|_{\infty}\geq|v_{i-1}\hat{c}+\hat{a}+v_{i+1}\hat{b}|\geq \hat{a}+\hat{b}+\hat{c}.
% \]
% Therefore
% \begin{equation}\label{eq:2D:boundT}
% \|\mathscr{M}^{-1}\|_{\infty}\leq\frac{1}{\hat{a}+\hat{b}+\hat{c}}.
% \end{equation}
%
% To prove \eqref{eq:2D:shishkinblocks} we just apply the bound \eqref{eq:2D:boundT} to matrices $\matAH$, $\matA$, and $\matAh$. For $\matAH$ we get,
% $$
%     |e_H + d_H| \|\matAH^{-1}\|_\infty \leq \frac{|e_H + d_H|}{a_H+b_H+c_H}
%     = \frac{|e_H + d_H|}{|e_H + d_H| + \beta} \leq 1,
% $$
% and the other inequalities in \eqref{eq:2D:shishkinblocks} follow analogously.
%
%
% Hence, we can apply Theorem~\ref{thm:2D:main2}.  It holds that
% $$
% \eta_{h,\infty}^{\klein{\min}} = \min \left\{ \frac{|d_h| \|\matAh^{-1}\|_{\infty}}
% {1-|e_h|\|\matAh^{-1}\|_{\infty}},  \frac{|d_h|\|\matAh^{-1}\|_{\infty}}
% {1-|d_h|\|\matAh^{-1}\|_{\infty}} \right\} =  \frac{|d_h|\|\matAh^{-1}\|_{\infty}}
% {1-|e_h|\|\matAh^{-1}\|_{\infty}} \leq 1,
% $$
% and
% $$
% \eta_{H,\infty}^{\klein{\min}}
% = \min \left\{\frac{|e_H|\|\matAH^{-1}\|_{\infty}}{1-|d_H| \|\matAH^{-1}\|_{\infty}},
% \frac{|e_H|\|\matAH^{-1}\|_{\infty} }{1-|e_H|\|\matAH^{-1}\|_{\infty}}\right\}
% = \frac{|e_H|\|\matAH^{-1}\|_{\infty} }{1-|e_H|\|\matAH^{-1}\|_{\infty}}
% \leq \left|\frac{e_H}{d_H}\right|.
% $$
% The convergence factor $\rho_{ij}$ can be bounded by
% \begin{eqnarray*}
% \left(\frac{\eta_{h,\infty}^{\klein{\min}} \|\matA^{-1}\matC\|_{\infty}}{1-\eta_{h,\infty}^{\klein{\min}}
% \|\matA^{-1}\matB\|_{\infty}}\;
% \frac{\eta_{H,\infty}^{\klein{\min}}\|\matA^{-1}\matB\|_{\infty}}{1-
% \eta_{H,\infty}^{\klein{\min}}\|\matA^{-1}\matC\|_{\infty}}\right)
% \leq \eta_{h,\infty}^{\klein{\min}}\eta_{H,\infty}^{\klein{\min}} \leq \left|\frac{e_H}{d_H}\right|
% \end{eqnarray*}
% and
% $$
% 	\| \T_{12} \| \leq
% 	\eta_{H,\klein{\infty}} \leq
% 	\left|\frac{e_H}{d_H}\right|,\qquad
% 	\| \T_{21} \| \leq \eta_{h,\klein{\infty}}^{\klein{\min}} \leq 1\,.
% $$

















\section{The Model Problem and its Shishkin Mesh Discretization}
% \label{2D:Problem}
%
% We consider the following convection-diffusion model problem with Dirichlet
% boundary conditions,
% %
% \begin{equation}\label{eq:2D:bvp2}
% -\epsilon  \Delta u + \alpha\cdot \nabla u+\beta u = f  \text{ in } \Omega=(0,1)\times(0,1),
% \quad \quad
% u=g\,\text{  on  }\, \Gamma=\partial\Omega.
% \end{equation}
% %
% Here the scalar-valued function $u(x,y)$ represents the concentration of a
% transported quantity, $\alpha(x,y)=[a_1(x,y),a_2(x,y)]^T$ the velocity field,
% $\epsilon$ the scalar diffusion parameter, and $\beta$ the scalar reaction
% parameter. We assume that on $\overline{\Omega}$ the components of the velocity
% field are bounded, that is, $a_1(x,y)\geq\alpha_1>0$ and
% $a_2(x,y)\geq\alpha_2>0$. Furthermore, we are interested in the
% \textit{convection dominated} case, i.e., the case when
% $\|\alpha\|\gg \epsilon >0$ in \eqref{eq:2D:bvp2}.
% By making the assumptions that $\alpha$, $\beta$, and $f$ are sufficiently
% smooth and that $\beta(x,y)-\frac{1}{2}\nabla \cdot \alpha(x,y)\geq C_0>0$ on
% $\overline{\Omega}$ for some constant $C_0$, we ensure that \eqref{eq:2D:bvp2} has
% a unique solution in the Sobolev space $H_0^1(\Omega)\cap H^2(\Omega)$ for all
% functions $f\in L^2(\Omega)$ \cite{FraLiuRooStyZho09}.
%
% We assume that the parameters of the problem, i.e. $\epsilon, \alpha, \beta$,
% $f$, $g$, are chosen so that the solution $u(x,y)$ has one boundary layer at
% $y=1$. A common approach for discretizing these types of problems is to resolve
% the boundary layers using a Shishkin mesh discretization, a scheme which uses
% pieceswise-uniform meshes that are constructed a priori. This technique has
% been described in detail, for example, in the survey
% article~\cite[Section~5]{Sty05} or \cite{KopOri10} and the
% book~\cite{MilOriShi96}. We therefore only state the facts that are relevant
% for our analysis.
%
% In order to obtain a satisfactoy approximation to the solution
% of~\eqref{eq:2D:bvp2}, we will discretize the two-dimensional domain using a mesh
% that is refined inside the layer; a very similar approach to the one used in
% the one-dimensional case (see \cite{EchLieSzyTic18}). To this end, we use a
% hybrid discretization procedure which combines the use of a uniform mesh in
% the $x$-direction with a Shishkin mesh in the $y$-direction.
%
% Let the positve real number $J\geq 3$ represent the number of intervals used in
% the homogenous mesh in the $x$-direction, and the even positive integer
% $N\geq 4$ denote the number of mesh intervals used for the Shishkin mesh in
% the $y$-direction. We define the transition parameter $\tau_y$, which will be
% used to specify where the mesh changes from coarse to fine in, by
% %
% \begin{equation}\label{eq:tau}
% \tau_y \equiv \min\left\{\frac12,2\frac{\epsilon}{\alpha_2}  \ln N\right\}.
% \end{equation}
% %
% The assumption $\alpha_2\gg \epsilon$ (equivalently $\epsilon \leq CN^{-1}$)
% leads to the value $\tau_y=2\frac{\epsilon}{\alpha_2}\ln N\ll 1$,
% so that the \emph{mesh transition point} $1-\tau_y$ will be very close to $y=1$.
% The use of this hybrid scheme will divide the domain $\Omega$ into two
% overlapping subdomains, $\overline{\Omega}=\Omega_{1}\cup\Omega_{2}$, where
% %
% \[
% \Omega_{1}=[0,1]\times[0,1-\tau_y],\quad\Omega_{2}=[0,1]\times[1-\tau_y,1],
% \]
% %
% this subdivision is shown in the left side of Figure~\ref{fig:2Ddomain}.
% %
% \begin{figure}[h!]
% \hspace*{-2em}
% \begin{minipage}[b]{0.5\textwidth}
% \begin{center}
% \begin{tikzpicture}[scale = 0.4]
% %%%% Rectangle:
% \draw (0,0) rectangle (10, 10);
% %%%% horizontal lines and $y$
% \node[left] at (0,10) {$1$};
% \draw (0,8) node[left] {$1-\tau_y$} -- (10,8);
% \node[left] at (0,0) {$0$};
% \node[below] at (0,0) {$0$};
% \node[below] at (10,0) {$1$};
% %%%% Black dots:
% \draw[black,fill=black] (0,8) circle (0.5ex);
% %%%% Tile labels
% \node at (5,4) {$\Omega_1$};
% \node at (5,9) {$\Omega_2$};
% \end{tikzpicture}
% \end{center}
% \end{minipage}
% %
% \hspace*{1em}
% %
% \begin{minipage}[b]{0.5\textwidth}
% \begin{center}
% \begin{tikzpicture}[scale = 0.4]
% %%%% Rectangle:
% \draw (0,0) rectangle (10, 10);
%
% %%%% horizontal lines and $y$
% \foreach \point in {2,4,6,8,8.5,9,9.5}
% \draw (0,\point) -- (10,\point);
% \node[left] at (0,10) {$y_N$};
% \node[left] at (0,8)  {$y_n$};
% \node[left] at (0,0)  {$y_0$};
%
% %%%% vertical lines and $x$:
% \foreach \point in {2,4,6,8}
% \draw (\point,0) -- (\point,10);
% \node[below] at (0,0) {$x_0$};
% \node[below] at (10,0) {$x_J$};
%
% %%%% Black dots:
% \draw[black,fill=black] (0,8) circle (0.5ex);
% %%%% Tile labels
% \draw (5,10) node[above] {$H_x$};
% \draw (10,2.5) node[right] {$H_y$};
% \draw (10,8.7) node[right] {$h_y$};
% \end{tikzpicture}
% \end{center}
% \end{minipage}
% %\includegraphics[width=0.9\linewidth]{figures/simpleShish2D3.pdf}
% \caption{Division of the domain and Shishkin mesh for equation \eqref{eq:2D:bvp2}
% with one outflow exponential layer.}
% \label{fig:2Ddomain}
% \end{figure}
%
% Let $n\equiv \frac{N}{2}$, if we denote the mesh width in the $x$-direction
% by $H_x$ and the mesh widths inside and outside the boundary layer in the
% $y$-direction by $h_y$ and $H_y$, i.e.,
% %
% \begin{equation}\label{eq:nHh}
% H_x:=\frac{1}{J},
% \;\;\quad
% h_y:=\frac{\tau_y}{n},
% \;\;\quad\mbox{and}\;\;\quad
% H_y:=\frac{( 1-\tau_y)}{n},
% \end{equation}
% then the $N+1$ nodes of the Shishkin mesh are given by
% \[
% \Omega^N=\{(x_i,y_j)\in\overline{\Omega}:i=0,\ldots,J; j=0,\ldots,N\},
% \]
% where
% %
% \[x_i\equiv iH_x,\;\mbox{for}\;\; i=0,\ldots,J,
% \quad \mbox{ and } \quad
% y_j\equiv
% \begin{cases} jH_y &\mbox{for}\;\; j=0,\ldots,n, \\
% 1-(N-j)h_y &\mbox{for}\;\; j=n+1,\ldots,N. \end{cases}\]
% %
% The mesh is constructed by drawing lines parallel to the coordinate axes
% through these mesh points; see the right side of Figure~\ref{fig:2Ddomain}.
% Here $x_0=y_0=0$ and $x_J=y_N=1$ so that the mesh consists of $K:=J-1$ interior
% nodes in the $x$-direction and $M:=N-1$ interior nodes in the $y$-direction,
% where the node $y_n$ is exactly at the transition point $1-\tau_y$.
% It is clear that the mesh widths on $\Omega_{1}$ satisfy $1/J\leq H_x\leq 2/J$
% and $1/N\leq H_y\leq 2/N$ so the mesh is coarse in this domain. On the other
% hand, $h_y$ is $\mathscr{O}(\epsilon N^{-1}\log(N))$, so on $\Omega_2$ the mesh
% is coarse in the $x$ direction and fine in the $y$ direction.  The ratio
% between the different mesh sizes in the $y$ direction is
% %
% \[\frac{h_y}{H_y}=\frac{\tau_y}{1-\tau_y}=\tau_y+{\cal O}(\tau_y^2) \ll 1.\]
%
% Using the standard upwind finite difference operators and the lexicographical
% line ordering of the unknowns of the hybrid mesh, the finite difference scheme
% yields a linear algebraic system $Au^{N}=f^{N}$ of the form \eqref{eq:2D:blockmat}
% with the block-tridiagonal matrix $A$ given by
% \begin{equation}\label{eq:back:2Dmatrix}
% A=\left[
%   \begin{array}{cccc|c|cccc}
%      A_{H} & B_{H}    &  &  &  &  &  &  &  \\
%      C_H &\ddots &\ddots  &  &  &  &  &  &  \\
%      &  \ddots& \ddots &  B_{H}   &  &  &  &  &  \\
%      &  &C_H & A_{H}  &  B_{H}   &  &  &  &  \\ \hline
%      &  &  & \hat{C}  & \hat{A}  & \hat{B}  &  &  &  \\ \hline
%      &  &  &  & C_{h}  & A_{h} & B_{h}  &  &  \\
%      &  &  &  &  & C_{h}  & \ddots & \ddots &  \\
%      &  &  &  &  &  & \ddots & \ddots & B_{h}  \\
%      &  &  &  &  &  &  & C_{h}  & A_{h}  \\
%   \end{array}
% \right]\;\in\;{\mathbb R}^{M^2\times M^2},
% \end{equation}
% where $M\equiv N-1$.
% The blocks $C_H$, $A_H$, $B_H$, etc., each of dimension $M\times M$, in
% \eqref{eq:back:2Dmatrix} are given by
% \begin{eqnarray}
% C_H&=&\text{diag}(\entryvS),\quad A_H=\text{tridiag}(\entryvW,\entryvC,\entryvE),
% \quad B_H=\text{diag}(\entryvN),\nonumber\\
% \hat{C}&=&\text{diag}(\entryzS), \qquad \,\,\hat{A}=\text{tridiag}(\entryzW,\entryzC,\entryzE),\qquad\quad\,\,\,\, \hat{B}=\text{diag}(\entryzN),\\
% C_h&=&\text{diag}(\entrywS),\quad\,\,\, A_h=\text{tridiag}(\entrywW,\entrywC,\entrywE), \quad \,\,\,\,B_h=\text{diag}(\entrywN),\nonumber
% \end{eqnarray}
% where the entries are given by
% \begin{scriptsize}
% \begin{align*}
% \hspace{-3em}
% 	&\entryvS  =  - \frac{\epsilon}{H_y^{2}}-\frac{\alpha_2}{H_y},&
%     &\entryvW  =  -\frac{\epsilon}{H_x^{2}}- \frac{\alpha_1}{H_x},&
%     &\entryvC  =  \frac{2\epsilon}{H_x^{2}} +\frac{2\varepsilon}{H_y^{2}} +\frac{\alpha_1}{H_x}+\frac{\alpha_2}{H_y}+\beta,
%     &\entryvE  =  -\frac{\epsilon}{H_x^{2}}, &
%     &\entryvN  =  - \frac{\epsilon}{H_y^{2}}, \\
%     &\entryzS  =  - \frac{2\epsilon}{H_y(H_y+h_y)}-\frac{\alpha_2}{H_y},&
%     &\entryzW  =  -\frac{\epsilon}{H_x^{2}} - \frac{\alpha_1}{H_x},&
%     &\entryzC  =  \frac{2\epsilon}{H_x^{2}} +\frac{2\varepsilon}{H_yh_y} +\frac{\alpha_1}{H_x}+\frac{\alpha_2}{H_y}+\beta,
%     &\entryzE  =  -\frac{\epsilon}{H_x^{2}}, &
%     &\entryzN  =  - \frac{2\epsilon}{h_y(H_y+h_y)},\label{eq:2D:upwind}\\
%     &\entrywS  =  - \frac{\epsilon}{h_y^{2}}-\frac{\alpha_2}{h_y},&
%     &\entrywW  =  -\frac{\epsilon}{H_x^{2}}- \frac{\alpha_1}{H_x},&
%     &\entrywC  =  \frac{2\epsilon}{H_x^{2}} +\frac{2\varepsilon}{h_y^{2}} +\frac{\alpha_1}{H_x}+\frac{\alpha_2}{h_y}+\beta,
%     &\entrywE  =  -\frac{\epsilon}{H_x^{2}}, &
%     &\entrywN  =  - \frac{\epsilon}{h_y^{2}}.
% \end{align*}
% \end{scriptsize}
% That is, our matrix takes the form
% \begin{equation}
% A=\left[
%   \begin{array}{c|c|c}
%              \matAHhat         & e_{m}\otimes\matBH   &                           \\ \hline
%     e_{m}^{\Tr}\otimes\hat{C}  &   \hat{A}            & e_{1}^{\Tr}\otimes\hat{B} \\ \hline
%                                & e_{1}\otimes\matCh   &        \matAhhat  \\
%   \end{array}
% \right],
% \end{equation}
% which has the same structure as \eqref{eq:2D:blockmat}.
%


% \section{Differences to 1D}
% \label{2D:Difference}

\ifnum\switch=1
% -----------------------------------------------------------------------------
% ------------------------------ Extended ToC ---------------------------------
% -----------------------------------------------------------------------------

This chapter is based on the following references: \cite{EchLieNab18, EchLieSzyTic18, EchLieTic19}.


\else
% -----------------------------------------------------------------------------
% --------------------------- Content of Chapter ------------------------------
% -----------------------------------------------------------------------------
Motivated by the challenge of understanding and analyzing the convergence of
the multiplicative Schwarz method for solving linear algebraic systems arising
from a ``simple'' and problem-specific one-dimensional model problem, various
results and contributions presented in this thesis are valid in a much broader
and general context. This concluding section provides a very brief discussion
on the consequences brought forth by the main results found in this
work and points out possibilities for further investigations that naturally
arise from the analysis and experiments present in this thesis.

% several contributions have been made to the mathematical theory
% as well  of the solution discretized
% convection-diffusion problems. Starting in Chapter~\ref{ch:1D}, we provide a
% convergence analysis of the multiplicative Schwarz method when used to solve
% one-dimensional problems. Motivated by the mathematical tools used to analyze
% such problems, we present the theory of block diagonal dominant matrices in
% Chapter~\ref{ch:BDiDo}. Finally, in Chapter~\ref{ch:2D}, we use the newly
% developed theory to analyze the convergence of the method for a general class
% of matrices whcih often appears in the discretization of second-order partial
% differential equations.

\medskip

In the context of Schwarz solution methods, our main contributions include
expressions for the convergence factor of the multiplicative Schwarz method
when used to solve linear algebraic systems with a special block structure arising from problems in one- and two-dimensions (see Corollary~\ref{cor:1D:rank.one} and
Lemma~\ref{lem:2D:genbounds}). The specific structure present in the
coefficient matrix is brought about when the finite difference or the finite
element method are used to discretize any second-order constant-coefficient elliptic partial differential equation which is posed on a very broad
and common domain decomposition context: that of a domain being subdivided into
two smaller subdomains with an overlap between them.
By making further assumptions on the system matrix, mainly that it
possesses a block-tridiagonal structure, we provide quantitative convergence
bounds for the error of the method in terms of the norms of the off-diagonal
blocks of the original system matrix (see Theorems~\ref{thm:2D:main} and
\ref{thm:2D:main2}). The provided bounds are valid from the first step of the
iteration process, presenting a grand improvement in contrast with the classic
convergence theory for multiplicative Schwarz methods, which is based on
asymptotic convergence rates and is not able to describe the transient phase of
the iteration scheme, where most problems like stagnation of the method might
occur. Furthermore, our analysis does not lean on any of the usual assumptions
put on the matrices needed to prove convergence in the classical domain
decomposition convergence theory, such as symmetry, or the $M$- or $H$-matrix
properties, making our approach much more general in its range of applications.

Contributions to the general area of preconditioning are also present in this
work. Results on the convergence theory of the preconditioned GMRES method,
given in the form of bounds on the residual norm of the iterates of the method,
are provided for the case when the multiplicative Schwarz method is used as a
preconditioner. The bounds are given in terms of the rank of the iteration
matrix of the multiplicative Schwarz method, which in the domain decomposition
context, we have proven to be of low-rank (see Lemma~\ref{lem:2D:powers}).
Moreover this result might be extended to the case when other fixed point
iteration methods are used as preconditioners in particular scenarios, since
the convergence of the GMRES method in a small number of steps is only
linked to the low rank structure of the iteration matrix of the method. In the
specific case presented in this thesis, bounding the rank of the iteration
matrix of the multiplicative Schwarz method allows us to prove rapid
convergence of the preconditioned GMRES method, an approach that can be useful
whenever the iteration matrices of other fixed point iteration methods posses a
low rank structure.

The specific domain decomposition setting used to present our theory, which
mimics the one fist studied by Schwarz at the end of the 19th century and
consists of only two overlapping subdomains, reduces the technicalities present
in the analysis needed to obtain the results, nevertheless, the arguments used
to construct the convergence theory can be applied in a recursive fashion. It is
easy to imagine each of the subdomains being subdivided once again into two
smaller subdomains and for this process to be repeated until a desired
finite number of subdivisions is reached in each of the original subdomains.
This, of course, has immediate potential in the implementation of the method for
parallel computing systems when a low memory storage prevents the modeling of
large domains in a single computing station.

In the specific case where the multiplicative Schwarz method is used to solve
systems coming from discretizations of the the convection-diffusion equation we
have shown that the approach of using a one- or two-dimensional Shishkin mesh
together with a finite difference approximation of the derivatives is very
effective in terms of the number of iterations that the method needs to
converge to an accurate solution, since in this context the iteration matrix
has low rank. Not only that, but our analysis clarifies an apparent
contradiction in the behavior of the method when used to solve these type of
problems, where we observe that  as the problem becomes harder
(we choose a smaller perturbation parameter) the convergence of the
multiplicative Schwarz method becomes faster and more effective
(see Theorems~\ref{thm:1D:upwind_conv} and \ref{thm:1D:central_conv} and
Corollary~\ref{cor:2D:conv-diff}). The effectivity of the bounds was
illustrated numerically on one- and two-dimensional convection-diffusion model
problems, showing that the bounds are extremely close to the actual errors
produced by the method in both cases, becoming more accurate as the problems
become more convection dominated.

The most general results presented in this work fall under the broad
study area of linear algebra and include a generalization of the
classical definition of block diagonal dominance of matrices given by Feingold
and Varga in~\cite{FeiVar62} (note that a very similar definition to the one
presented in this thesis appeared on the same year in
\cite{BenEvaHamLupSla17}), which brings new understanding of the concept of
diagonal dominance when dealing with operators with a block structure (see Definitions~\ref{def:BDiDo:bdd} and \ref{def:App:colbdd}). In the
classical definitions of block diagonal dominance, the blocks are ``treated as scalars'', however in our approach the block diagonal dominance is based on the
influence of the blocks as matrices, i.e., the action of the block is taken
into account not just their norms. This new definition and its consequences
have proven to be useful tools in the analysis of the multiplicative Schwarz
method and might very well be useful in analyzing other iterative methods,
specially in the context of fixed point iteration methods. Moreover, the generalization of block diagonal dominance provided in
this work implies new results in the spectral theory of eigenvalues, for
example, it infers new eigenvalues inclusion sets for general block matrices
which are potentially tighter that the classical ones given by the previous
definitions of diagonal dominance or the Gershgorin's Circle Theorem (see
Corollary~\ref{cor:BDiDo:inclusion}).

Furthermore, based on our definition we established upper and lower bounds and
decay rates on the block norms of the inverse of block diagonally dominant
block tridiagonal matrices (see Theorems~\ref{thm:BDiDo:blockbounds} and
\ref{thm:BDiDo:iterblockbounds} and Corollary~\ref{cor:BDiDo:decay}), which are
a direct generalization of the bounds presented in \cite{Nabben99} for the
scalar case. The bounds may be extremely useful for the creation of
preconditioners for large linear systems when the system matrices are block
tridiagonal. The off-diagonal decay of the block norms of its inverse matrix
implied by the bounds allows for the creation of approximate inverses by
neglecting blocks that fall bellow a desired norm threshold.

\medskip

Although the results provided in this thesis focus on the study and
generalization of simple model problems, natural extensions of our approach to
more general settings are still possible. We will now briefly mention a number
of generalizations and alternative applications to our analysis that may be
explored in future research.

Our analysis stems from on a very specific domain decomposition problem setting
which translates into solving linear systems with coefficient matrices
exhibiting a specific block structure. The problem setting which was chosen is
the simplest one possible and the main assumption that guided our analysis was
the fact that in each of the two local subdomains, we use the same number of
unknowns. This assumption was helpful in reducing technicalities present in our
analysis, however, by relaxing this and other different constraints, we can
envision the same approach being applied to matrices $\A$ with more general
structures.

It would be only natural to apply our approach in a setting where the local
subdomains have a different number of unknowns. This would be reflected in a
coefficient matrix $\A$ with blocks $\matAHhat$ and $\matAhhat$ of variable
size in \eqref{eq:2D:blockmat}. An analysis of the problem in this more general
context would be filled with more technicalities, however it would be certainly
possible to carry out, specially when the sizes of the matrices are multiples
of $N$, i.e. for matrices $\matAHhat\in\mathbb{R}^{Ns\times Ns}$ and
$\matAhhat\in\mathbb{R}^{Nt\times Nt}$. It is also easy to think of applying our
analysis to the solution of discretized of PDEs with variable coefficients.
This change would be reflected in the structure of the matrices $\matAHhat$ and
$\matAhhat$ by exhibiting general tridiagonal blocks of the form
\begin{equation}\label{eqn:tridiaggeneral}
\matAHhat=\mathrm{tridiag}(C_{\klein{H},i},A_{\klein{H},i},B_{\klein{H},i})
\quad\text{and}\quad
\matAhhat=\mathrm{tridiag}(C_{\klein{h},i},A_{\klein{h},i},B_{\klein{h},i}),
\end{equation}
instead of tridiagonal Toeplitz matrices of the form \eqref{eq:2D:tridiag}. The
analysis would still be possible since the theory of block diagonal dominance presented in Chapter~\ref{ch:BDiDo} used to obtain the
bounds allows for relaxing this constraint. A generalization of
\ref{thm:2D:main2} to $\A$ with blocks \eqref{eqn:tridiaggeneral} would require
that the conditions \eqref{eq:2D:dominant} hold in every block row, and then
analogously to \eqref{eq:2D:etah}--\eqref{eq:2D:etaH}, every block row in
$\matAHhat$ or $\matAhhat$ would give a parameter $\eta_{H,i}$
or $\eta_{h,i}$, respectively. For convection-diffusion problems, the structure of the matrix would also be affected by the number of boundary layers present in the problem. We would have more than one overlap region in the domain and the iteration matrixes would present different more complicated nonzero block structures affecting the rank of the iteration matrix. However, preliminary numerical experiments show that the iteration matrices still possess a low numerical rank (close to $N$). Thus, we believe that it is still analyzable
along the lines of Section~\ref{2D:bounds:structure}.

% \medskip
%
% The author believes that the analysis presented in this thesis opens up the possibility of developing a general convergence theory of fixed point iteration methods based on the structure of the system matrices where a systematic analysis of different structures leading to various convergence results. Going even further, this new theory could be extended, like it is done in this work, to the convergence of GMRES when a general fixed point iteration method is used as a preconditioner.


\fi

\ifnum\switch=1
%% -----------------------------------------------------------------------------
% ------------------------------ Extended ToC ---------------------------------
% -----------------------------------------------------------------------------

Brief introductory commentary.

\section{Scope and goals}
The main question that this thesis aims to answer.

\section{Summary of main results}
The main findings of this Thesis are...

\section{Content outline}
Organization of the Thesis

\else
% -----------------------------------------------------------------------------
% --------------------------- Content of Chapter ------------------------------
% -----------------------------------------------------------------------------

In a wide range of applications in the general field of scientific computing it
is important to solve linear algebraic systems of the form
\begin{equation}
\label{eq:int:linsys}
\A\u=\f,
\end{equation}
% $$
% \mathcal{A}\mathcal{u}=\mathcal{f}
% $$
% $$
% \A\u=\f
% $$
% $$
% Au=f
% $$
where the matrix
%$\A\in\C^{N\times N}$
$\A\in\mathbb{C}^{N\times N}$ in \eqref{eq:int:linsys} is nonnormal,
ill-conditioned and possesses a specific block sparse structure.
% and the right hand side $b\in\C^N$.
Very often, matrices of this class arise when discretizing boundary value
problems (BVPs); mathematical models which try to describe the asymptotic
(steady-state) solution of a particular partial differential equation (PDE)
posed on a specific domain. In particular they appear in the discretization
process of BVPs that describe the behavior of fluid flow, like the ones
modelled by the steady-state convection-diffusion BVP:
\begin{equation}
\label{eq:int:BVP}
-\epsilon \Delta u + \om\cdot\nabla u + \beta u = f
\text{ in } \Omega, % = (0,1)\times(0,1),\quad
\quad u = g,
\text{ on } \partial\Omega.
\end{equation}
% \begin{equation}
% \hspace*{2em}
% \begin{cases}
% -\epsilon \Delta u + \om \cdot \nabla u+\beta u = f, &
% \text{ in }\; \Omega\in\Re^n\\
% %\;\;u|_{\partial \Omega}=g. &
% \hspace{2.96cm} u = g, &
% \text{ on }\; \partial \Omega.
% \end{cases}
% \label{eq:int:BVP}
% \end{equation}
% Here the scalar-valued function $u(x,y)$ represents the concentration of a
% transported quantity, $\alpha(x,y)=[a_1(x,y),a_2(x,y)]^T$ the velocity field,
% $\epsilon$ the scalar diffusion parameter, and $\beta$ the scalar reaction
% parameter. We assume that on $\overline{\Omega}$ the components of the %
% velocity field are bounded, that is, $a_1(x,y)\geq\alpha_1>0$ and
% $a_2(x,y)\geq\alpha_2>0$. Furthermore, we are interested in the
% \textit{convection dominated} case, i.e., the case when
% $\|\alpha\|\gg \epsilon >0$ in \eqref{eq:int:BVP}.
% By making the assumptions that $\alpha$, $\beta$, and $f$ are sufficiently
% smooth and that $\beta(x,y)-\frac{1}{2}\nabla \cdot \alpha(x,y)\geq C_0>0$ on
% $\overline{\Omega}$ for some constant $C_0$, we ensure that \eqref{eq:int:BVP}
% has a unique solution in the Sobolev space $H_0^1(\Omega)\cap H^2(\Omega)$
% for all functions $f\in L^2(\Omega)$ \cite{FraLiuRooStyZho09}.
In the PDE of problem \eqref{eq:int:BVP}, the scalar-valued function $u$ is
commonly interpreted as the concentration of a transported quantity, $\om$ the
velocity field or \textit{wind} where the concentration is transported,
$\epsilon$~the scalar diffusion coefficient, and $\beta$ the scalar reaction
coefficient~-~both measures of the amount of diffusion and
production~/~destruction of the concentration throughout the domain $\Omega$.

In most applications and relevant cases of convection-diffusion BVPs, it
is important to study the \textit{convection dominated} regime, i.e., the cases
when $\|\om\|\gg \epsilon >0$ in \eqref{eq:int:BVP}, leading to what is known in
literature as a \textit{singularly perturbed} boundary value problem and where
$\epsilon$ is known as the \textit{perturbation parameter}.
One possible physical interpretation of this type of problems, as described by Elman, Silvester and Wathen in \cite{ElmSilWat14}, may be the
following: the unknown function $u$ may represent the concentration of a
pollutant, being transported, or ``convected'', along a river moving at velocity
$\om$, while being subject to diffusive and reactive effects. In this context,
the solution to the BVP would describe the final concentration of the pollutant
at each point of the riverbed.

Singularly perturbed convection-diffusion BVPs of type \eqref{eq:int:BVP} often
exhibit the presence of \textit{boundary layers}, small regions of the domain
$\Omega$ where the solution $u$ exhibits a sharp change in its gradient. In
turn, the existence of boundary layers presents a challenge for numerical
methods to finding an accurate numerical representation of $u$, and usually
require special discretization techniques in order to guarantee the stability
of a numerical solution method \cite{Mor96}. A very popular approach is to use finite
difference methods for approximating the derivatives of \eqref{eq:int:BVP} on a
piecewise equidistant mesh, known as a \textit{Shihskin mesh}, which
decomposes the domain into subregions with different resolutions; typically
allowing to emphasize computational attention in the region of interest inside
of the boundary layers present in each particular problem; see e.g., \cite{Sty13}. A general overview
of these types of problems, their difficulties and solution approaches can be
found, e.g., in the excellent survey article~\cite{Sty05}.

The decomposition of the domain into various subregions, caused by the use of a
Shishkin mesh to discretize the domain, is reflected on the resulting
discretized convection-diffusion operators~$\A$, by exhibiting a particular
block sparse structure. Consequently, the property of block-sparsity in the
entries of the operator suggests the implementation of iterative solution
methods when solving linear systems of the form \eqref{eq:int:linsys} with such
coefficient matrices \cite{Saa03}. In particular, the use of domain decomposition methods
like the multiplicative Schwarz method seems to be a natural choice of solution
approach for these problems and, indeed, its efficiency is corroborated by
numerical experiments (see Figures~\ref{fig:1D:MSM.N198.eps8}-\ref{fig:1D:MSM.N10002.eps4} which show the convergence of the multiplicative
Schwarz method and compare them to Figures~\ref{fig:back:GMRES.N198.eps4}--\ref{fig:back:GMRES.N198.eps4} which show the convergence of the
unpreconditioned GMRES method). Most of the work presented in this thesis will
concern with the analysis of the multiplicative Schwarz method for solving
systems of type \eqref{eq:int:linsys} coming from one- and two-dimensional
finite difference discretizations of problems of type~\eqref{eq:int:BVP}.

\medskip

The (algebraic) multiplicative Schwarz method, which is often
called the alternating Schwarz method (see~\cite{Gan08} for a historical
survey), is a stationary iterative method for solving large
and sparse linear algebraic systems of the form \eqref{eq:int:linsys}. In each
step of the method, the current iterate is multiplied by
an iteration matrix that is the  product of several factors, where each factor
corresponds an inversion of only  a restricted part of the matrix. In the
context of interest of this thesis, i.e., the numerical solution of discretized
convection-diffusion problems, the restrictions of the matrix correspond to
different parts of the computational domain subdivided by the Shishkin
mesh. This motivates the name ``local solve'', which is a popular term used to
describe each of these factors and is also used in a purely algebraic setting.
The method is for the most part used as a preconditioner for a Krylov subspace
method such as GMRES and many of its convergence results have been presented in
that context; see, e.g., the treatment in the books~\cite{DolJolNat15, TosWid05}, and many references therein. When the method is considered from an
algebraic point of view, as we do in this work, it is commonly treated as a s
solution method; see, e.g., \cite{BenFroNabSzy01}. The convergence theory for
the multiplicative Schwarz method is well established for important matrix
classes including symmetric positive definite matrices and nonsingular
$M$-matrices~\cite{BenFroNabSzy01}, symmetric indefinite
matrices~\cite{FroNabSzy08,FroSzy14}, and $H$-matrices~\cite{BruPedSzy04}. The
derivation of convergence results for these matrix classes is usually based on
splittings of $\A$ and no systematic convergence theory exists however for
general nonsymmetric matrices.

Several authors have previously applied the alternating (or multiplicative)
Schwarz method to the continuous problem \eqref{eq:int:BVP} based on the
partitioning of the domain into overlapping subdomains, and subsequently
discretized by introducing uniform meshes on each subdomain; see,
e.g.,~\cite{FarHegMilOriShi00, FarShi00, MacMilOriShi00, MacMilOriShi01,MacOriShi00, MacOriShi02,MilOriShi96}.
However, as clearly explained in~\cite{MacOriShi02}, significant numerical
problems including very slow convergence and accumulation of errors (up to the
point of non-convergence of the numerical solution) can occur when
layer-resolving mesh transition points are used in this setup. These problems
are avoided in our approach, since we first discretize and then apply
the multiplicative Schwarz method to the linear algebraic system. To the best
of our knowledge, this approach has not been studied in the literature so far.

For problems in one spacial dimension, studied in Chapter~\ref{ch:1D}, the
structure of the coefficient matrices exhibits a tridiagonal structure and the
main mathematical tool used in our analysis exploits the fact that such
discrete operators are \emph{diagonally dominant}. For problems in higher spatial dimensions, studied
in Chapter~\ref{ch:2D}, the discretized operator exhibits a block tridiagonal
structure. In order to perform an analysis analogous to the one-dimensional
case, we generalize the property of diagonal dominance from the scalar case to
the case where the matrices possess a block structure and develop a new
mathematical theory of \emph{block diagonal dominant} matrices. The theory is
presented in Chapter~\ref{ch:BDiDo} and it is general enough that it can be
applied to any matrix with a block structure, however it seems to be
particularly useful for matrices coming from discretizations of PDEs.

As mentioned before, the system matrices we study in this work are
nonsymmetric,  nonnormal, ill-conditioned, and in particular not in one of the
classes considered in~\cite{BenFroNabSzy01,BruPedSzy04,FroNabSzy08,FroSzy14}.
Moreover, our derivations are not based on matrix splittings, but on the
off-diagonal decay of the matrix inverses, which in turn is implied by diagonal
dominance. From a broader point of view our results show why a convergence
theory for the multiplicative Schwarz method for ``general'' matrices will most
likely remain elusive: Even in the simple model problem considered
in Chapter~\ref{ch:1D} and in \cite{EchLieSzyTic18}, the convergence of the
method strongly depends on the problem parameters and on the chosen
discretization, and while the method rapidly converges in some cases, it
diverges in others.

\section{Scope and Goals}
The scope of this thesis aims to provide an analysis of the convergence
behavior of the multiplicative Schwarz method when it is used to solve linear
systems arising from special finite difference discretizations of singularly
perturbed convection-diffusion problems posed on a Shishkin mesh. We restrict our attention to the cases where the domain $\Omega$ is one- or
two-dimensional, and we analyze the method both as an algebraic solution
approach as well as a preconditioner for the GMRES method.

The analysis presented in this work brings an understanding on why this
solution technique is so effective for solving problems arising from the
Shishkin mesh discretizations. We do this by providing convergence bounds for
the norm of the error generated by the method at each iteration step.
Moreover, the convergence bounds provided in this thesis, in the paper
\cite{EchLieNab18}, and in the manuscript \cite{EchLieTic19}, shed light on
an apparent contradiction: If the continuous problem becomes more difficult
(a smaller diffusion coefficient is chosen), then the convergence of the
multiplicative~Schwarz method for the discretized problem becomes faster.

The mathematical tools developed in this work to achieve its main goal are,
however, more general. The theory of block diagonal dominance of matrices
presented in Chapter~\ref{ch:BDiDo}, with the bounds and eigenvalue inclusion
sets resulting from our analysis, does not only apply to operators coming
from discretizations of BVPs but is applicable to any matrix
$\A\in\mathbb{C}^{N\times N}$ with a block structure.

% In short, the main questions that this thesis aims to answer are:
% %\td{\textbf{write}: reformulate main questions of this thesis}
%
% \begin{itemize}
% \item[•] How does the \emph{error} of the multiplicative Schwarz method
% behave when solving linear systems \eqref{eq:int:linsys} coming from finite
% difference discretizations of singularly perturbed convection-diffusion
% problems of the form \eqref{eq:int:BVP} in conjunction with the use of
% appropriate Shishkin fitted meshes?
% % \end{itemize}
% %
% % \begin{itemize}
% \item[•] How does the \emph{residual} of the GMRES method behave when solving
% linear systems \eqref{eq:int:linsys} coming from finite difference
% discretizations of singularly perturbed convection-diffusion problems of the
% form \eqref{eq:int:BVP} in conjunction with the use of appropriate Shishkin
% fitted meshes, when the multiplicative Schwarz method is used as a
% preconditioner?
% \end{itemize}
%
% The solution of discretized convection diffusion problems is typically
% reformulated into inverting nonsymmetric nonnormal matrices. Thus,
% mathematically, we are trying to answer the more general question:
% how does the multiplicative Schwarz method (as an algebraic solution approach
% and as a preconditioner for GMRES) behave when it is used to solve linear
% systems with nonsymmetric nonnormal matrices with a special block structure?
% In order to give an answer to these questions, will adhere to the following
% guidelines:
% %
% \begin{itemize}
% \item[•]
% Given that there is no general theory to treat all nonsymmetric
% nonnormal matrices and all iterative numerical methods, we focus on the
% matrices arising from a simple one-dimensional model problem.
% %\td{\textbf{note}: see comment in reference [39] of Liesen \& Strako{\^ s}: block splittings and CG}
%
% \item[•] Using the problem specific results obtained in the one-dimensional
% model problem (exploiting the structure of the discrete operators),
% we develop general theoretical tools (theory of block diagonal dominance of
% matrices) which allow for the generalization of the results to prove and
% describe the convergence behavior of the iterative method in higher
% dimensional cases.
%
% \item[•]  Once the specific one-dimensional case is analytically proven, we use
% the generalizability of model problems and the newly developed tools to answer
% the questions for higher dimensional cases; see for example the interesting
% discussion about the nature of model problems in~\cite{KraParSte83}.
% \end{itemize}

\section{Outline and Summary of Main Results}

We begin by introducing a short review of the theoretical background material
needed to understand the results presented in this thesis in
Chapter~\ref{ch:back}. An experienced reader in these topics, might use this
chapter for reference and directly visit Chapters~\ref{ch:1D}~--~\ref{ch:2D} which encompass the main findings of this thesis: results for
one-dimensional model problems, results for general matrices, and results for
two-dimensional model problems. Finally, Chapter~\ref{ch:end} provides a
brief discussion and outlook on possible continuations of this work. The
appendix \ref{ch:matlab} provides instructions for obtaining the computational
code in order to perform and reproduce the numerical experiments presented
throughout this thesis. In the following we provide a brief summary of the
results obtained in each of the main chapters of this thesis:
% As mentioned above, the main findings of this thesis are subdivided into
% the three chapters:
%\td{\textbf{write}: summary of the main findings of this work is still missing.}
% \begin{itemize}
%
% \item[•]
\paragraph{Chapter~\ref{ch:1D}:} In this chapter, we analyze the
convergence of the multiplicative Schwarz method applied to nonsymmetric linear
algebraic systems obtained from discretizations of one-dimensional singularly
perturbed convection-diffusion equations by upwind and central finite
differences on a Shishkin mesh. Using the algebraic structure of the Schwarz
iteration matrices we derive bounds on the infinity norm of the error that are
valid from the first step of the iteration. Our bounds for the upwind scheme
prove rapid convergence of the multiplicative Schwarz method for all relevant
choices of parameters in the problem. The analysis for the central difference
is more complicated, since the submatrices that occur are nonsymmetric and
sometimes even fail to be $M$-matrices. Our bounds still prove the convergence
of the method for certain parameter choices.

\paragraph{Chapter~\ref{ch:BDiDo}:} Here we generalize the bounds
on the inverses of diagonally dominant matrices obtained in~\cite{Nabben99}
from scalar to block tridiagonal matrices. Our derivations are based on a
generalization of the classical condition of block diagonal dominance of
matrices given by Feingold and Varga in~\cite{FeiVar62}. Based on this
generalization, which was recently presented in \cite{EchLieNab18} and a
similar definition appearing first in \cite{BenEvaHamLupSla17}, we also derive
a variant of the Gershgorin Circle Theorem for general block matrices which can
provide tighter spectral inclusion regions than those obtained by Feingold and
Varga.

\paragraph{Chapter~\ref{ch:2D}:} Finally, we analyze the convergence
of the multiplicative Schwarz method applied to linear algebraic systems with
matrices having a special block structure that arises, for example, when a
partial differential equation is posed and discretized on a domain that
consists of two subdomains with an overlap. This is a basic situation in the
context of domain decomposition methods. Again, our analysis is based on the
algebraic  structure of the Schwarz iteration matrices, and we derive error
bounds that are based on the block diagonal dominance of the given system
matrix. Our analysis does not assume that the system matrix is symmetric
(positive definite), or has the $M$- or $H$-matrix property. Our approach in
this chapter significantly generalizes the analysis for a special
one-dimensional model problem treated in Chapter~\ref{ch:1D}.
% \end{itemize}

% Here we analyze the convergence of the multiplicative Schwarz method for
% linear algebraic systems with matrices of the form
% %
% \begin{equation}\label{eq:blockmat}
% \mathcal{A}=\left[
%   \begin{array}{ccc}
%              \matAHhat       & e_{m}\otimes \matBH   &    0            \\
%     e_{m}^{\Tr}\otimes \matC  &   A      & e_{1}^{\Tr}\otimes \matB \\
%                    0         & e_{1}\otimes\matCh &        \matAhhat  \\
%   \end{array}
% \right]\;\in\;\R^{N(2m+1)\times N(2m+1)},
% \end{equation}
% %
% with $\matAHhat, \matAhhat \in \R^{Nm\times Nm}$,
% $\matA, \matB, \matC, \matBH, \matCh \in \R^{N\times N}$, and
% the canonical basis vectors $e_1,e_m\in\R^{m}$. We will usually think of
% $\matAHhat, \matAhhat \in \R^{Nm\times Nm}$ as matrices consisting of $m$
% blocks of size $N\times N$. After deriving general expressions for the norms
% of the multiplicative Schwarz iteration matrices for systems of the form
% \eqref{eq:linsys}--\eqref{eq:blockmat}, we derive actual error bounds only for
% the case when the blocks $\matAHhat$ and $\matAhhat$ of ${\cal A}$ are block
% tridiagonal.
%
% Such matrices arise naturally when a differential equation is posed and
% discretized inside a domain $\Omega$ that is divided into two subdomains
% $\Omega_1$ and $\Omega_2$ with an overlap. In this context the first $m+1$
% block rows in the matrix $\mathcal{A}$ correspond to the unknowns in the
% domain $\Omega_1$, the last $m+1$ block rows correspond to the unknowns in
% the domain $\Omega_2$, and the middle block row corresponds to the unknowns
% in the overlap. The underlying assumption here is that in each of the two
% subdomains we have the same number of unknowns. This assumption is made for
% simplicity of our exposition. Extensions to other block sizes are possible,
% but would require even more technicalities; see our discussion in
% Section~\ref{sec:conclusions}.

% \section{Content Outline}


\fi % end of if statement regarding content shown
